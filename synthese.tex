\documentclass[8pt]{article}

\usepackage[utf8]{inputenc}
\usepackage{multicol}
\usepackage[margin=0.6in]{geometry}

\title{Synthèse bibliographique - Opinion Mining et Analyse de sentiments}
\author{Florian Martin et Thibault Théologien}
\date{\today}

\begin{document}
  \maketitle
  \begin{multicols}{2}

% Introduction
\par Avec l’avènement du Web 2.0 (Réseaux sociaux, blogs, forums de discussion, ...), une grande quantité d’information est disponible en ligne. Elles peuvent fournir des indications précieuses dans divers domaines du fait qu’elles reflètent l’opinion de communautés sur divers sujets. De plus, du fait de cette quantité il devient rapidement difficile d’identifier celles qui sont pertinentes.
\par Des recherches sont menées activement afin de permettre de mieux organiser ces informations \cite{ressource21} \cite{ressource2} \cite{ressource5}. L’analyse de sentiment s’est démarquée ces dernières années avec l’apparition d’un grand nombre de sites demandant l’avis de leurs clients sur divers produits et de discussions de groupe.
\par Nous nous intéresserons par la suite aux principes de l’analyse de sentiment, puis nous étudierons les divers outils et méthodes disponibles. Nous verrons ensuite les principaux supports utilisés pour les recherches ainsi que les divers domaines d’application. Enfin, nous serons amenés à parler des limites actuelles liées à ce domaine de recherche.

% Principes
\par Le principal objectif de l’analyse de sentiments ou du minage d’opinion (OM) est de déterminer la polarité d’une information \cite{ressource6}, c’est-à-dire trouver si l’ensemble a un sens plutôt positif, négatif ou neutre.
\par Pour cela, il faut dans un premier temps de traiter l’information pour qu’elle ait une forme simple à étudier. Il peut être nécessaire de filtrer (Retrait des liens, ...), segmenter, alléger des mots n’ayant pas de caractère informatif (Exemples : un, le, la...) ou mettre sous la forme de n-grammes (augmente la précision du modèle) \cite{ressource10}.
\par Une estimation de l’orientation sémantique peut ensuite être effectuée. Elle est obtenue en analysant la polarité et l’objectivité des mots. Ce dernier peut être obtenue grâce à un classifieur entraîné, dont l’apprentissage a été effectué sur un grand nombre de données dont le sentiment dégagé est connu. Ces informations sont aujourd’hui  simples à collecter \cite{ressource10} du fait de la multitude de réseaux sociaux (Twitter \cite{ressource11}, Facebook \cite{ressource27}, ...) et de sites proposant des avis sur divers produits (e-commerce, ...), ce qui permet d’étudier des corpus variés sur différents domaines.
\par Pour obtenir des informations utiles une fois la polarité obtenue il est possible pour des systèmes de recommandation ou pour établir des statistique de classer les données. Nous pouvons déterminer la crédibilité d’une donnée par rapport aux autres \cite{ressource28} ou encore déterminer sa pertinence au sujet \cite{ressource18}. Il est également possible d’établir un classement  simplement grâce à leur polarité \cite{ressource4}.
\par Dans le but de vérifier les résultats, des données annotées manuellement sont généralement utilisées \cite{ressource30}. Cependant des techniques d’annotation automatiques existent \cite{ressource19}.

% Données
\par Aujourd'hui une énorme quantité d’information disponible en ligne provenant de sources diverses et variées qui permettent si elles sont analysées de fournir des statistiques sur le ressenti de différentes communautés sur une multitude de sujets.
\par Parmi les sources les plus fréquemment utilisées et fournissant le plus d’informations nous retrouvons les réseaux sociaux comme Facebook \cite{ressource27} ou encore Twitter \cite{ressource10} \cite{ressource11} \cite{ressource14} \cite{ressource17} \cite{ressource19} \cite{ressource22} \cite{ressource23} \cite{ressource24} \cite{ressource28} qui voit un grand nombre de recherches utiliser ses données pour divers domaines du fait de la brièveté et la quantité de texte à disposition publiquement. Nous trouvons également les plateformes proposant des systèmes de commentaires comme YouTube \cite{ressource30} \cite{ressource22} où les utilisateurs réagissent à des vidéos. Enfin, sont utilisés les sites permettant aux utilisateurs de commenter et donner des notes sur des produits comme IMDb \cite{ressource1} qui permet d’évaluer des films et séries.


% Méthodes
\par Ce domaine étant en constante évolution, de nombreuses méthodes existent, certaines exploitant des techniques liées à la fouille de données telles que le machine learning \cite{ressource28} \cite{ressource1}, les forêts aléatoires \cite{ressource18}, les arbres à noyaux \cite{ressource30} \cite{ressource17} et bien d’autres types de classifieur \cite{ressource4} \cite{ressource9} \cite{ressource10} \cite{ressource15}; d’autres sont plus étroitement liées à l’analyse sémantique des données \cite{ressource16} \cite{ressource11} \cite{ressource12} \cite{ressource7}.
De nouvelles méthodes telles que VADER \cite{ressource26}, OPINE \cite{ressource9}, SentiStrength \cite{ressource22} et bien d’autres apparaissent régulièrement \cite{ressource17} \cite{ressource16} \cite{ressource13}, tandis que d’autres comme SENTIWORDNET \cite{ressource12} \cite{ressource7} sont progressivement améliorées.

% Domaines d'application
Les domaines d’application de l’OM sont nombreux \cite{ressource20}. En politique, il peut être utilisé afin de prédire le résultat d'élections ou de sondages \cite{ressource11} \cite{ressource14}, pour déterminer l’approbation de la population quant aux actions du gouvernement \cite{ressource14} ou encore pour déterminer “l’humeur générale” d’une population \cite{ressource27} \cite{ressource25} \cite{ressource24}.
D’un point de vue économique ou marketing, les techniques d’OM permettent d’évaluer la qualité d’un produit \cite{ressource18} \cite{ressource9} \cite{ressource8} ou encore d’obtenir un avis résumant l’ensembles des avis utilisateur \cite{ressource3}

% Limites
\par Les méthodes actuelles sont confrontées à divers problèmes \cite{ressource8}: l’ambivalence des phrases et les comparaisons rendant difficile une détection correcte des sentiments exprimés; la distribution correspondant aux avis étant biaisée (il a été observé que la majorité des utilisateurs donnent un avis positif); les phrases courtes manquant d’information sur les sentiments exprimés; les systèmes de notation souvent biaisés (les perceptions sont propres aux individus).
\end{multicols}
\par D’autres problèmes peuvent être liés à l’utilisation des diverses plateformes qui permettent de donner un avis \cite{ressource5}. Des robots et utilisateurs peuvent fausser le ressenti général sur un sujet en postant plusieurs fois un même avis. Il n’existe aujourd’hui aucune solution automatique et concluante palliant à ces inconvénients.
\par Il a été observé également que les sentiments pouvaient êtres exprimés différemment selon le domaine étudié. Les méthodes actuelles ont donc des difficultés à s’adapter. Des solutions ont été étudiées pour minimiser cet impact et mesurer l’adaptabilité d’un classifieur \cite{ressource15}.
\par De part le caractère encore expérimental de ce domaine, et le nombre de méthodes existantes, il peut être complexe de déterminer quelle algorithme utiliser selon le contexte. Des études comparatives ont donc été menées \cite{ressource29} \cite{ressource26}, et des outils développés \cite{ressource29} \cite{ressource19}.


% Conclusion
\par Nous avons vu au cours de cette synthèse que les méthodes d’analyse de sentiment et de minage d’opinion ont de nombreuses application, ajoutant une plus-value informative aux données présentes en grande quantité sur internet.
\par Cependant les méthodes actuelles ont des difficultés à s’adapter aux changements de domaines et sont sensibles aux différents abus possibles par les utilisateurs sur les diverses plateformes, ce qui fausse en partie les résultats obtenus.


  \newpage

  \bibliographystyle{unsrt}
  \bibliography{synthese}
  \nocite{*}
\end{document}
