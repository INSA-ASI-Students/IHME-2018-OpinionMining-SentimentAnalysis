\documentclass[8pt]{article}

\usepackage[utf8]{inputenc}
\usepackage{multicol}
\usepackage[margin=0.6in]{geometry}

\title{Synthèse bibliographique - Opinion Mining et Analyse de sentiments}
\author{Florian Martin et Thibault Théologien}
\date{\today}

\begin{document}
  \maketitle
  \begin{multicols}{2}
    % Introduction
    \par Avec l’avènement du Web 2.0 (Réseaux sociaux, blogs, forums de discussion, ...), une grande quantité d’information est disponible en ligne. Elles peuvent fournir des indications précieuses dans divers domaines du fait qu’elles reflètent l’opinion de communautés sur divers sujets. De plus, du fait de cette quantité il devient rapidement difficile d’identifier celles qui sont pertinentes. Des recherches sont menées activement afin de permettre de mieux organiser ces informations. Nous pouvons noter les recherches sur la classification selon le sujet, mais l’analyse de sentiment c’est démarquée ses dernières années avec l’apparition rapide de sites demandant l’avis de leurs clients sur divers produits et les discussions de groupe.
    \par Nous nous intéresserons dans cette synthèse bibliographique aux principes de l’analyse de sentiment et du minage d’opinion, puis nous étudierons les divers outils et méthodes disponibles à ce jour. Nous verrons ensuite les principaux supports de communication utilisés pour les recherches ainsi que les divers domaines d’application des diverses méthodes. Enfin, nous serons amenés à parler des limites actuelles liées à ce domaine de recherche.

  \end{multicols}

  \newpage

  \bibliographystyle{unsrt}
  \bibliography{synthese}
  \nocite{*}
\end{document}
