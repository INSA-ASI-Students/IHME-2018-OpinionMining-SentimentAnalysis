\documentclass[8pt]{article}

\usepackage[utf8]{inputenc}
\usepackage{multicol}
\usepackage[margin=0.6in]{geometry}

\title{Synthèse bibliographique - Opinion Mining et Analyse de sentiments}
\author{Florian Martin et Thibault Théologien}
\date{\today}

\begin{document}
  \maketitle
  \begin{multicols}{2}

% Introduction
\par Avec l’avènement du Web 2.0 (Réseaux sociaux, blogs, forums de discussion, ...), une grande quantité d’information est disponible en ligne. Elles peuvent fournir des indications précieuses dans divers domaines du fait qu’elles reflètent l’opinion de communautés sur divers sujets. De plus, du fait de cette quantité il devient rapidement difficile d’identifier celles qui sont pertinentes.
\par Des recherches sont menées activement afin de permettre de mieux organiser ces informations. L’analyse de sentiment s’est démarquée ces dernières années avec l’apparition d’un grand nombre de sites demandant l’avis de leurs clients sur divers produits et de discussions de groupe.
\par Nous nous intéresserons par la suite aux principes de l’analyse de sentiment, puis nous étudierons les divers outils et méthodes disponibles. Nous verrons ensuite les principaux supports utilisés pour les recherches ainsi que les divers domaines d’application. Enfin, nous serons amenés à parler des limites actuelles liées à ce domaine de recherche.

% Principes
\par Le principal objectif de l’analyse de sentiments ou du minage d’opinion (OM) est de déterminer la polarité d’une information {Recognizing contextual polarity in phrase-level sentiment analysis}, c’est-à-dire trouver si l’ensemble a un sens plutôt positif, négatif ou neutre.
\par Pour cela, il faut dans un premier temps de traiter l’information pour qu’elle ait une forme simple à étudier. Il peut être nécessaire de filtrer (Retrait des liens, ...), segmenter, alléger des mots n’ayant pas de caractère informatif (Exemples : un, le, la...) ou mettre sous la forme de n-grammes (augmente la précision du modèle) {Twitter as a Corpus for Sentiment Analysis and Opinion Mining}.
\par Une estimation de l’orientation sémantique peut ensuite être effectuée. Elle est obtenue en analysant la polarité et l’objectivité des mots. Ce dernier peut être obtenue grâce à un classifieur entraîné, dont l’apprentissage a été effectué sur un grand nombre de données dont le sentiment dégagé est connu. Ces informations sont aujourd’hui  simples à collecter {Twitter as a Corpus for Sentiment Analysis and Opinion Mining} du fait de la multitude de réseaux sociaux (Twitter{Predicting Elections with Twitter: What 140 Characters Reveal about Political Sentiment}, Facebook{An Unobtrusive Behavioral Model of “Gross National Happiness"}, ...) et de sites proposant des avis sur divers produits (e-commerce, ...), ce qui permet d’étudier des corpus variés sur différents domaines.
\par Pour obtenir des informations utiles une fois la polarité obtenue il est possible pour des systèmes de recommandation ou pour établir des statistique de classer les données. Nous pouvons déterminer la crédibilité d’une donnée par rapport aux autres{Credibility ranking of tweets during high impact events} ou encore déterminer sa pertinence au sujet{Estimating the helpfulness and economic impact of product reviews: Mining text and reviewer characteristics}. Il est également possible d’établir un classement  simplement grâce à leur polarité{Thumbs up or thumbs down? Semantic orientation applied to unsupervised classification of reviews}.
\par Dans le but de vérifier les résultats, des données annotées manuellement sont généralement utilisées{Opinion Mining on YouTube}. Cependant des techniques d’annotation automatiques existent{Enhanced Sentiment Learning Using Twitter Hashtags and Smileys}.

% Données
\par Aujourd'hui une énorme quantité d’information disponible en ligne provenant de sources diverses et variées qui permettent si elles sont analysées de fournir des statistiques sur le ressenti de différentes communautés sur une multitude de sujets.
\par Parmis les sources les plus fréquemments utilisées et fournissant le plus d’informations nous retrouvons les réseaux sociaux comme Facebook {An Unobtrusive Behavioral Model of “Gross National Happiness"} ou encore Twitter {Twitter as a Corpus for Sentiment Analysis and Opinion Mining}{Predicting Elections with Twitter: What 140 Characters Reveal about Political Sentiment}{From tweets to polls: Linking text sentiment to public opinion time series}{Sentiment analysis of Twitter data}{Enhanced Sentiment Learning Using Twitter Hashtags and Smileys}{Sentiment strength detection for the social web}{Sentiment Knowledge Discovery in Twitter Streaming Data}{Temporal Patterns of Happiness and Information in a Global Social Network : Hedonometrics and Twitter}{Credibility ranking of tweets during high impact events} qui voit un grand nombre de recherches utiliser ses données pour divers domaines du fait de la brièveté et la quantité de texte à disposition publiquement. Nous trouvons également les plateformes proposant des systèmes de commentaires comme YouTube{Opinion Mining on YouTube}{Sentiment strength detection for the social web} où les utilisateurs réagissent à des vidéos. Enfin, sont utilisés les sites permettant aux utilisateurs de commenter et donner des notes sur des produits comme IMDb{Thumbs up? Sentiment classification using machine learning techniques} qui permet d’évaluer des films et séries.


% Méthodes
\par Ce domaine étant en constante évolution, de nombreuses méthodes existent, certaines exploitant des techniques liées à la fouille de données telles que le machine learning {Credibility ranking of tweets during high impact events} {Thumbs up? Sentiment classification using machine learning techniques}, les forêts aléatoires {Estimating the helpfulness and economic impact of product reviews: Mining text and reviewer characteristics}, les arbres à noyaux {Opinion Mining on YouTube} {Sentiment analysis of Twitter data} et bien d’autres types de classifieur {Thumbs up or thumbs down? Semantic orientation applied to unsupervised classification of reviews} {Extracting product features and opinions from reviews} {Twitter as a Corpus for Sentiment Analysis and Opinion Mining} {Biographies, bollywood, boom-boxes and blenders: Domain adaptation for sentiment classification}; d’autres sont plus étroitement liées à l’analyse sémantique des données {Lexicon-based methods for sentiment analysis} {Predicting Elections with Twitter: What 140 Characters Reveal about Political Sentiment} {SENTIWORDNET 3.0: An Enhanced Lexical Resource for Sentiment Analysis and Opinion Mining} {SENTIWORDNET: A high-coverage lexical resource for opinion mining}.
De nouvelles méthodes telles que VADER {Extracting product features and opinions from reviews}, OPINE {Extracting product features and opinions from reviews}, SentiStrength {Sentiment strength detection for the social web} et bien d’autres apparaissent régulièrement {Sentiment analysis of Twitter data} {Lexicon-based methods for sentiment analysis} {Measuring praise and criticism: Inference of semantic orientation from association}, tandis que d’autres comme SENTIWORDNET {SENTIWORDNET 3.0: An Enhanced Lexical Resource for Sentiment Analysis and Opinion Mining} {SENTIWORDNET: A high-coverage lexical resource for opinion mining} sont progressivement améliorées.

% Domaines d'application
Les domaines d’application de l’OM sont nombreux. En politique, il peut être utilisé afin de prédire le résultat d'élections ou de sondages {Predicting Elections with Twitter: What 140 Characters Reveal about Political Sentiment} {From tweets to polls: Linking text sentiment to public opinion time series}, pour déterminer l’approbation de la population quant aux actions du gouvernement {From tweets to polls: Linking text sentiment to public opinion time series} ou encore pour déterminer “l’humeur générale” d’une population {An Unobtrusive Behavioral Model of “Gross National Happiness"} {Measuring the Happiness of Large-Scale Written Expression: Songs, Blogs, and Presidents} {Temporal Patterns of Happiness and Information in a Global Social Network : Hedonometrics and Twitter}.
D’un point de vue économique ou marketing, les techniques d’OM permettent d’évaluer la qualité d’un produit {Estimating the helpfulness and economic impact of product reviews: Mining text and reviewer characteristics} {Extracting product features and opinions from reviews} {Mining the peanut gallery: Opinion extraction and semantic classification of product reviews} ou encore d’obtenir un avis résumant l’ensembles des avis utilisateur {Mining and summarizing customer reviews}

% Limites
\par Les méthodes actuelles sont confrontées à divers problèmes {Mining the peanut gallery: Opinion extraction and semantic classification of product reviews}: l’ambivalence des phrases et les comparaisons rendant difficile une détection correcte des sentiments exprimés; la distribution correspondant aux avis étant biaisée (il a été observé que la majorité des utilisateurs donnent un avis positif); les phrases courtes manquant d’information sur les sentiments exprimés; les systèmes de notation souvent biaisés (les perceptions sont propres aux individus).
\par D’autres problèmes peuvent être liés à l’utilisation des diverses plateformes qui permettent de donner un avis {Sentiment analysis and opinion mining}. Des robots et utilisateurs peuvent fausser le ressenti général sur un sujet en postant plusieurs fois un même avis. Il n’existe aujourd’hui aucune solution automatique et concluante palliant à ces inconvénients.
\par Il a été observé également que les sentiments pouvaient êtres exprimés différemment selon le domaine étudié. Les méthodes actuelles ont donc des difficultés à s’adapter. Des solutions ont été étudiées pour minimiser cet impact et mesurer l’adaptabilité d’un classifieur {Biographies, bollywood, boom-boxes and blenders: Domain adaptation for sentiment classification}.
\par De part le caractère encore expérimental de ce domaine, et le nombre de méthodes existantes, il peut être complexe de déterminer quelle algorithme utiliser selon le contexte. Des études comparatives ont donc été menées {Comparing and combining sentiment analysis methods} {VADER : A Parsimonious Rule - based Model for Sentiment Analysis of Social Media Text}, et des outils développés {Comparing and combining sentiment analysis methods} {Enhanced Sentiment Learning Using Twitter Hashtags and Smileys}.


% Conclusion
\par Nous avons vu au cours de cette synthèse que les méthodes d’analyse de sentiment et de minage d’opinion ont de nombreuses application, ajoutant une plus-value informative aux données présentes en grande quantité sur internet.
\par Cependant les méthodes actuelles ont des difficultés à s’adapter aux changements de domaines et sont sensibles aux différents abus possibles par les utilisateurs sur les diverses plateformes, ce qui fausse en partie les résultats obtenus.


  \end{multicols}

  \newpage

  \bibliographystyle{unsrt}
  \bibliography{synthese}
  \nocite{*}
\end{document}
